%% 
%% Copyright 2007-2020 Elsevier Ltd
%% 
%% This file is part of the 'Elsarticle Bundle'.
%% ---------------------------------------------
%% 
%% It may be distributed under the conditions of the LaTeX Project Public
%% License, either version 1.2 of this license or (at your option) any
%% later version.  The latest version of this license is in
%%    http://www.latex-project.org/lppl.txt
%% and version 1.2 or later is part of all distributions of LaTeX
%% version 1999/12/01 or later.
%% 
%% The list of all files belonging to the 'Elsarticle Bundle' is
%% given in the file `manifest.txt'.
%% 

%% Template article for Elsevier's document class `elsarticle'
%% with numbered style bibliographic references
%% SP 2008/03/01
%%
%% 
%%
%% $Id: elsarticle-template-num.tex 190 2020-11-23 11:12:32Z rishi $
%%
%%
\documentclass[preprint,12pt]{elsarticle}

%% Use the option review to obtain double line spacing
%% \documentclass[authoryear,preprint,review,12pt]{elsarticle}

%% Use the options 1p,twocolumn; 3p; 3p,twocolumn; 5p; or 5p,twocolumn
%% for a journal layout:
%% \documentclass[final,1p,times]{elsarticle}
%% \documentclass[final,1p,times,twocolumn]{elsarticle}
%% \documentclass[final,3p,times]{elsarticle}
%% \documentclass[final,3p,times,twocolumn]{elsarticle}
%% \documentclass[final,5p,times]{elsarticle}
%% \documentclass[final,5p,times,twocolumn]{elsarticle}

%% For including figures, graphicx.sty has been loaded in
%% elsarticle.cls. If you prefer to use the old commands
%% please give \usepackage{epsfig}

%% The amssymb package provides various useful mathematical symbols
\usepackage{amssymb}
%% The amsthm package provides extended theorem environments
%% \usepackage{amsthm}

%% The lineno packages adds line numbers. Start line numbering with
%% \begin{linenumbers}, end it with \end{linenumbers}. Or switch it on
%% for the whole article with \linenumbers.
%% \usepackage{lineno}

\journal{Nuclear Physics B}

\begin{document}

\begin{frontmatter}

%% Title, authors and addresses

%% use the tnoteref command within \title for footnotes;
%% use the tnotetext command for theassociated footnote;
%% use the fnref command within \author or \address for footnotes;
%% use the fntext command for theassociated footnote;
%% use the corref command within \author for corresponding author footnotes;
%% use the cortext command for theassociated footnote;
%% use the ead command for the email address,
%% and the form \ead[url] for the home page:
%% \title{Title\tnoteref{label1}}
%% \tnotetext[label1]{}
%% \author{Name\corref{cor1}\fnref{label2}}
%% \ead{email address}
%% \ead[url]{home page}
%% \fntext[label2]{}
%% \cortext[cor1]{}
%% \affiliation{organization={},
%%             addressline={},
%%             city={},
%%             postcode={},
%%             state={},
%%             country={}}
%% \fntext[label3]{}

\title{Title of Your Manuscript}

%% use optional labels to link authors explicitly to addresses:
%% \author[label1,label2]{}
%% \affiliation[label1]{organization={},
%%             addressline={},
%%             city={},
%%             postcode={},
%%             state={},
%%             country={}}
%%
%% \affiliation[label2]{organization={},
%%             addressline={},
%%             city={},
%%             postcode={},
%%             state={},
%%             country={}}

\author[inst1]{Author One}

\affiliation[inst1]{organization={Department One},%Department and Organization
            addressline={Address One}, 
            city={City One},
            postcode={00000}, 
            state={State One},
            country={Country One}}

\author[inst2]{Author Two}
\author[inst1,inst2]{Author Three}

\affiliation[inst2]{organization={Department Two},%Department and Organization
            addressline={Address Two}, 
            city={City Two},
            postcode={22222}, 
            state={State Two},
            country={Country Two}}

\begin{abstract}
%% Text of abstract
Lorem ipsum dolor sit amet, consectetur adipiscing elit, sed do eiusmod tempor incididunt ut labore et dolore magna aliqua. Ut enim ad minim veniam, quis nostrud exercitation ullamco laboris nisi ut aliquip ex ea commodo consequat. Duis aute irure dolor in reprehenderit in voluptate velit esse cillum dolore eu fugiat nulla pariatur. Excepteur sint occaecat cupidatat non proident, sunt in culpa qui officia deserunt mollit anim id est laborum.
\end{abstract}

%%Graphical abstract
% \begin{graphicalabstract}
% \includegraphics{grabs}
% \end{graphicalabstract}

%%Research highlights
\begin{highlights}
\item Research highlight 1
\item Research highlight 2
\end{highlights}

\begin{keyword}
%% keywords here, in the form: keyword \sep keyword
keyword one \sep keyword two
%% PACS codes here, in the form: \PACS code \sep code
\PACS 0000 \sep 1111
%% MSC codes here, in the form: \MSC code \sep code
%% or \MSC[2008] code \sep code (2000 is the default)
\MSC 0000 \sep 1111
\end{keyword}

\end{frontmatter}


\section{Introduction}
\label{sec:introduction}

The advent of the internet and widespread use of digital devices have sparked a profound shift in connectivity and data creation, leading to an era marked by a rapid growth in data. This period is characterised by the extensive expansion of data, which presents difficulties for traditional data processing systems and necessitates inventive methods in data architecture \cite{AtaeiACIS,AtaeiBigDataEnvirons}. The vast amount, variety, and rapid generation of data in the current digital environment necessitate innovative solutions, particularly in the field of Big Data (BD).

Data needs have dramatically evolved, transitioning from basic business intelligence (BI) functions, like generating reports for risk management and compliance, to incorporating machine learning across various organisational facets \cite{ataei2023towards}. These range from product design with automated assistants to personalised customer service and optimised operations.  Also, as machine learning becomes more popular, application development needs to change from rule-based, deterministic models to more flexible, probabilistic models that can handle a wider range of outcomes and need to be improved all the time with access to the newest data. This evolution underscores the need to reevaluate and simplify our data management strategies to address the growing and diverse expectations placed on data.

Currently, the success rate of BD projects is low. Recent surveys have identified the fact that current approaches to big data do not seem to be effectively addressing these expectations. According to a survey conducted by \citeauthor{DataBricksSurvey}, only 13\% of organisations are highly successful in their data strategy. Additionally, a report by NewVantage Partners reveals that only 24\% of organisations have successfully converted to being data-driven, and a measly 30\% have a well-established big data strategy. These observations, additionally corroborated by research conducted by McKinsey & Company (analytics2016age) and Gartner (Nash), emphasise the difficulties of successfully using big data in the industry. These difficulties include the lack of a clear understanding of how to extract value from data, the challenge of integrating data from multiple sources, data architecture, and the need for skilled data analysts and scientists. 

\section{Background}
\label{sec:background}
% Provide an overview of the foundational concepts and technologies pertinent to the BD reference architecture.

\section{Related Work}
\label{sec:related_work}
% Discuss existing research in the domain, highlighting gaps that your work aims to address.

\section{Why Reference Architectures}
\label{sec:why_reference_architectures}
% Justify the need for reference architectures in the context of Big Data and how they can address existing challenges.

\section{Software and System Requirements}
\label{sec:software_and_system_requirements}
% Detail the software and system requirements necessary to implement and evaluate the proposed architecture.

\section{Theory}
\label{sec:theory}
% Explain the theoretical underpinnings that guide the development of the reference architecture.

\section{Artifact}
\label{sec:artifact}
% Describe the design and development of the reference architecture artifact, focusing on its structure and components.

\section{Discussion}
\label{sec:discussion}
% Reflect on the findings from the evaluation, discussing their implications, limitations, and the relevance to existing and future research.

\section{Conclusion}
\label{sec:conclusion}
% Summarize the main contributions of your paper, its practical implications, and suggest directions for future research.




\appendix




 \bibliographystyle{elsarticle-num} 
 \bibliography{bibfile}

%% else use the following coding to input the bibitems directly in the
%% TeX file.

% \begin{thebibliography}{00}

% %% \bibitem{label}
% %% Text of bibliographic item

% \bibitem{}

% \end{thebibliography}
\end{document}
\endinput
%%
%% End of file `elsarticle-template-num.tex'.