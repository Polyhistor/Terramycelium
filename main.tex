%% 
%% Copyright 2007-2020 Elsevier Ltd
%% 
%% This file is part of the 'Elsarticle Bundle'.
%% ---------------------------------------------
%% 
%% It may be distributed under the conditions of the LaTeX Project Public
%% License, either version 1.2 of this license or (at your option) any
%% later version.  The latest version of this license is in
%%    http://www.latex-project.org/lppl.txt
%% and version 1.2 or later is part of all distributions of LaTeX
%% version 1999/12/01 or later.
%% 
%% The list of all files belonging to the 'Elsarticle Bundle' is
%% given in the file `manifest.txt'.
%% 

%% Template article for Elsevier's document class `elsarticle'
%% with numbered style bibliographic references
%% SP 2008/03/01
%%
%% 
%%
%% $Id: elsarticle-template-num.tex 190 2020-11-23 11:12:32Z rishi $
%%
%%
\documentclass[preprint,12pt]{elsarticle}

%% Use the option review to obtain double line spacing
%% \documentclass[authoryear,preprint,review,12pt]{elsarticle}

%% Use the options 1p,twocolumn; 3p; 3p,twocolumn; 5p; or 5p,twocolumn
%% for a journal layout:
%% \documentclass[final,1p,times]{elsarticle}
%% \documentclass[final,1p,times,twocolumn]{elsarticle}
%% \documentclass[final,3p,times]{elsarticle}
%% \documentclass[final,3p,times,twocolumn]{elsarticle}
%% \documentclass[final,5p,times]{elsarticle}
%% \documentclass[final,5p,times,twocolumn]{elsarticle}

%% For including figures, graphicx.sty has been loaded in
%% elsarticle.cls. If you prefer to use the old commands
%% please give \usepackage{epsfig}

%% The amssymb package provides various useful mathematical symbols
\usepackage{amssymb}
%% The amsthm package provides extended theorem environments
%% \usepackage{amsthm}

%% The lineno packages adds line numbers. Start line numbering with
%% \begin{linenumbers}, end it with \end{linenumbers}. Or switch it on
%% for the whole article with \linenumbers.
%% \usepackage{lineno}

\journal{Nuclear Physics B}

\begin{document}

\begin{frontmatter}

%% Title, authors and addresses

%% use the tnoteref command within \title for footnotes;
%% use the tnotetext command for theassociated footnote;
%% use the fnref command within \author or \address for footnotes;
%% use the fntext command for theassociated footnote;
%% use the corref command within \author for corresponding author footnotes;
%% use the cortext command for theassociated footnote;
%% use the ead command for the email address,
%% and the form \ead[url] for the home page:
%% \title{Title\tnoteref{label1}}
%% \tnotetext[label1]{}
%% \author{Name\corref{cor1}\fnref{label2}}
%% \ead{email address}
%% \ead[url]{home page}
%% \fntext[label2]{}
%% \cortext[cor1]{}
%% \affiliation{organization={},
%%             addressline={},
%%             city={},
%%             postcode={},
%%             state={},
%%             country={}}
%% \fntext[label3]{}

\title{Terramycelium: A Domain-driven Reference Architecture for Big Data Systems}

%% use optional labels to link authors explicitly to addresses:
%% \author[label1,label2]{}
%% \affiliation[label1]{organization={},
%%             addressline={},
%%             city={},
%%             postcode={},
%%             state={},
%%             country={}}
%%
%% \affiliation[label2]{organization={},
%%             addressline={},
%%             city={},
%%             postcode={},
%%             state={},
%%             country={}}

\author[inst1]{Author One}

\affiliation[inst1]{organization={Department One},%Department and Organization
            addressline={Address One}, 
            city={City One},
            postcode={00000}, 
            state={State One},
            country={Country One}}

\author[inst2]{Author Two}
\author[inst1,inst2]{Author Three}

\affiliation[inst2]{organization={Department Two},%Department and Organization
            addressline={Address Two}, 
            city={City Two},
            postcode={22222}, 
            state={State Two},
            country={Country Two}}

\begin{abstract}
%% Text of abstract
Lorem ipsum dolor sit amet, consectetur adipiscing elit, sed do eiusmod tempor incididunt ut labore et dolore magna aliqua. Ut enim ad minim veniam, quis nostrud exercitation ullamco laboris nisi ut aliquip ex ea commodo consequat. Duis aute irure dolor in reprehenderit in voluptate velit esse cillum dolore eu fugiat nulla pariatur. Excepteur sint occaecat cupidatat non proident, sunt in culpa qui officia deserunt mollit anim id est laborum.
\end{abstract}

%%Graphical abstract
% \begin{graphicalabstract}
% \includegraphics{grabs}
% \end{graphicalabstract}

%%Research highlights
\begin{highlights}
\item Research highlight 1
\item Research highlight 2
\end{highlights}

\begin{keyword}
%% keywords here, in the form: keyword \sep keyword
keyword one \sep keyword two
%% PACS codes here, in the form: \PACS code \sep code
\PACS 0000 \sep 1111
%% MSC codes here, in the form: \MSC code \sep code
%% or \MSC[2008] code \sep code (2000 is the default)
\MSC 0000 \sep 1111
\end{keyword}

\end{frontmatter}


\section{Introduction}
\label{sec:introduction}

The advent of the internet and widespread use of digital devices have sparked a profound shift in connectivity and data creation, leading to an era marked by a rapid growth in data. This period is characterised by the extensive expansion of data, which presents difficulties for traditional data processing systems and necessitates inventive methods in data architecture \cite{AtaeiACIS,AtaeiBigDataEnvirons}. The vast amount, variety, and rapid generation of data in the current digital environment necessitate innovative solutions, particularly in the field of Big Data (BD).

Data needs have dramatically evolved, transitioning from basic business intelligence (BI) functions, like generating reports for risk management and compliance, to incorporating machine learning across various organisational facets \cite{ataei2023towards}. These range from product design with automated assistants to personalised customer service and optimised operations.  Also, as machine learning becomes more popular, application development needs to change from rule-based, deterministic models to more flexible, probabilistic models that can handle a wider range of outcomes and need to be improved all the time with access to the newest data. This evolution underscores the need to reevaluate and simplify our data management strategies to address the growing and diverse expectations placed on data.

Currently, the success rate of BD projects is low. Recent surveys have identified the fact that current approaches to big data do not seem to be effectively addressing these expectations. According to a survey conducted by \cite{DataBricksSurvey}, only 13\% of organisations are highly successful in their data strategy. Additionally, a report by NewVantage Partners reveals that only 24\% of organisations have successfully converted to being data-driven, and a measly 30\% have a well-established big data strategy. These observations, additionally corroborated by research conducted by McKinsey \& Company (analytics2016age) and Gartner (Nash), emphasise the difficulties of successfully using big data in the industry. These difficulties include the lack of a clear understanding of how to extract value from data, the challenge of integrating data from multiple sources, data architecture, and the need for skilled data analysts and scientists. 

Without a well-established big data strategy, companies may struggle to navigate these challenges and fully leverage the potential of their data. One effective artefact to overcome some of these challenges is Reference Architectures (RAs) \cite{Cloutier2010}. RAs extract the essence of the practice as a series of patterns and architectural constructs and manifest it through high-level semantics. This allows stakeholders to refrain from reinventing the wheel and instead focus on utilising existing knowledge and best practices to harness the full potential of their data. While there are various BD RAs available to help practitioners design their BD systems, these RAs are overly cetnralised, lack attention to cross-cutting concerns such as privacy, security, and metadata, and may not effectively handle the proliferation of data sources and consumers.

To this end, this study presents TerrMycelium, a distributed RA designed specifically for BD systems with a focus on domain-driven design. TerrMycelium seeks to surpass the constraints of current RAs by utilising domain-driven and distributed approaches derived from contemporary software engineering. This method aims to improve the ability of BD systems to scale, be maintained, and evolve, surpassing the constraints of traditional monolithic data architectures.

The paper is structured as follows: Section~\ref{sec:background} provides an overview of the foundational concepts and technologies pertinent to BD reference architecture, aiming to forge a conceptual framework that is required for this paper. An overview of the existing research on the topic is presented in Section~\ref{sec:related_work}. The significance of reference architectures in the context of big data is explored in Section~\ref{sec:why_reference_architectures}. Section~\ref{sec:software_and_system_requirements} details the software and system requirements necessary for implementing the proposed architecture. Section~\ref{sec:theory} delves into the theoretical foundation underpinning the challenges in contemporary big data systems. The design and development of the TerrMycelium artifact are described in Section~\ref{sec:artifact}. Section~\ref{sec:discussion} examines the evaluation findings, their implications, limitations, and relevance to existing and future research. Finally, Section~\ref{sec:conclusion} summarizes the main contributions of the study, its practical implications, and suggests directions for future research.


\section{Background}
\label{sec:background}

This section provides foundational definitions essential for comprehending the nuances of the research. This chapter aims to create the conceptual framework necessary to understand the terminology used in the thesis.

\subsection{What is Big Data?}

To define BD within the scope of this research, various academic definitions have been examined. \citet{Kaisler2013} define BD as ``the amount of data which is beyond technology’s capability to store, manage and process efficiently''.
\citet{Srivastava2018} state that BD pertains to ``the use of large data sets to handle the collection or reporting of data that serves various recipients in decision making''.

\citet{Sagiroglu2013} describe BD as ``a term for massive data sets having large, more varied and complex structure with the difficulties of storing, analyzing and visualizing for further processes or results''. 

Drawing from these definitions, BD in this research is conceptualised as the endeavour to discern patterns from vast amounts of data for the objectives of advancement, governance, and predictive analysis in domain-specific applications.

\subsection{The Value of Big Data}\label{sec:The Value of Big Data}

The significance and value derived from BD remain pronounced \cite{ataei2022state}. Extensive discussions on the concept permeate reports, statistics, researches, and conferences \cite{Chen2012}. Notably, prominent companies like Google, Facebook, Netflix, and Amazon have propelled this momentum with substantial investments in BD initiatives \cite{Rada2017}.

A compelling illustration of the tangible benefits that BD offers can be seen in the Netflix Prize recommender system. This system capitalized on a diverse array of data sources, including user queries, ratings, search terms, and various demographic indicators \cite{Amatriain2013}. By implementing BD-powered recommendation algorithms, Netflix not only achieved a considerable increase in TV series consumption but also observed certain series experiencing up to a fourfold surge in viewership \cite{Amatriain2013}.


In a healthcare context, the Taiwanese government adeptly merged its national health insurance database with customs and immigration datasets as part of a BD strategy \cite{wang2020response}. The resulting real-time alerts during clinical visits, informed by clinical symptoms and travel history among other factors, facilitated proactive identification of potential COVID-19 cases. Such strategic data-driven initiatives significantly bolstered Taiwan's effectiveness in managing the epidemic.

In the realm of energy exploration, Shell harnesses BD to optimise the decision-making process and reduce exploration costs \cite{Marr2016}. By uploading and comparing data from various drilling sites globally, decisions pivot towards locations that mirror those with confirmed abundant resources. Prior to BD's integration, identifying energy resources presented formidable challenges. Traditional exploration methods, which relied heavily on deciphering waves of energy travelling through the earth's crust, were not only error-prone but also exorbitantly expensive and time-intensive.

Similarly, Rolls Royce capitalises on BD's potential by collecting intricate performance data from sensors fitted on its aircraft products \cite{Marr2016}. Such data, transmitted wirelessly, provides insights into key operational phases, from take-off to maintenance. Leveraging this wealth of information, Rolls Royce can more accurately detect degradation, enhance diagnostic and prognostic accuracy, and effectively reduce false positives.

\subsection{Reference Architectures}

RAs have emerged as pivotal elements in contemporary system development, guiding the construction, maintenance, and evolution of increasingly complex systems \cite{Cloutier2010}. They offer a clear depiction of the essential components of a system and the interactions necessary to realize overarching objectives. This clarity fosters the creation of manageable modules, each addressing distinct aspects of complex problems, and provides a high-level platform for stakeholders to engage, contribute, and collaborate.

The significance of RAs in IT is underscored by the success of widely adopted technologies like OAuth \cite{OATH} and ANSI-SPARC architecture \cite{ANSI}, which have their origins in well-structured RAs. These RAs not only define the qualities of a system but also shape its evolution. While every system inherently possesses an architecture, RAs distinguish themselves by focusing on more abstract qualities and higher levels of abstraction. They aim to capture the essence of practice and integrate well-established patterns into cohesive frameworks, encompassing elements, properties, and interrelationships.

The significance of RAs in BD is multifaceted, encompassing aspects like communication, complexity control, knowledge management, risk mitigation, fostering future architectural visions, defining common ground, enhancing understanding of BD systems, and facilitating further analysis.

\subsection{Microservices and Decentralised, Distributed Architectures}

Microservices architecture, representing an evolution in software engineering, involves structuring applications as a collection of loosely coupled services \cite{bucchiarone2020microservices}. This approach, emerging from the broader concept of Service Oriented Architectures (SOA), focuses on developing small, independently deployable modules that collaborate to form a comprehensive application. As \citet{Newman2015} elucidates, microservices enhance scalability, facilitate continuous deployment, and foster a more agile development environment. They enable teams to develop, deploy, and scale parts of a system independently, thus improving overall system resilience and facilitating rapid adaptation to changing demands.

Decentralised and distributed architectures are integral to the modern computing landscape, characterised by systems spread across multiple nodes, often in different geographic locations. This architectural style, as highlighted by \citet{Richards2015}, mitigates the limitations of traditional monolithic structures, offering enhanced scalability, fault tolerance, and flexibility. In distributed systems, data and processing are dispersed across multiple nodes, which interact with each other to perform tasks, as discussed by \citet{Coulouris2005}. Decentralisation in this context implies the lack of a single controlling node, instead opting for a more democratic and resilient network structure.

The convergence of microservices within these architectures represents a progressive step in software engineering. It reflects a move towards systems that are not only distributed in nature but also modular and adaptable. This architectural approach aligns well with contemporary demands for systems that are scalable, resilient, and capable of leveraging the distributed nature of modern computing environments. The adoption of microservices in decentralised, distributed architectures heralds a new era in software development, where flexibility, scalability, and resilience are paramount.




\section{Related Work}
\label{sec:related_work}
% Discuss existing research in the domain, highlighting gaps that your work aims to address.

\section{Why Reference Architectures}
\label{sec:why_reference_architectures}
% Justify the need for reference architectures in the context of Big Data and how they can address existing challenges.

\section{Software and System Requirements}
\label{sec:software_and_system_requirements}
% Detail the software and system requirements necessary to implement and evaluate the proposed architecture.

\section{Theory}
\label{sec:theory}
% Explain the theoretical underpinnings that guide the development of the reference architecture.

\section{Artifact}
\label{sec:artifact}
% Describe the design and development of the reference architecture artifact, focusing on its structure and components.

\section{Discussion}
\label{sec:discussion}
% Reflect on the findings from the evaluation, discussing their implications, limitations, and the relevance to existing and future research.

\section{Conclusion}
\label{sec:conclusion}
% Summarize the main contributions of your paper, its practical implications, and suggest directions for future research.




\appendix




\bibliographystyle{elsarticle-harv}

 \bibliography{bibfile}

\end{document}
\endinput
%%
%% End of file `elsarticle-template-num.tex'.